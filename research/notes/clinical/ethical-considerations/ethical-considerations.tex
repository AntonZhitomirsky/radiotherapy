\documentclass[11pt]{article}

\usepackage[margin=1in]{geometry}

\usepackage{hyperref}

\usepackage{biblatex} %Imports biblatex package
\addbibresource{../../../source/bibliography.bib} %Import the bibliography file

\setlength{\parindent}{0pt}

\title{Ethical considerations of working with Medical data}
\author{Anton Zhitomirsky}

\begin{document}

\maketitle

\section{Introduction}

In this project many samples of woman's 3D anatomy is used to train a network to delineate structures within. Such tools are created to accelerate clinitian's abilities to accurately, algorithmically and automatically provide planning target volumes to administer radiotherapy treatment. On paper, this is a great cause for using the population's private information to fast track treatments for recurrent gynaecological cancers (RGC).

However, the following document will detail the concerns on the flip side of this project which may jopardise the right to privacy of subjects and clinitian's apprihensive attitude towards assistant segmentation tools.

\section{Why should medical data be private?}

\subsection{General Anonymity}

The right to privacy is referred to as a shared human right. This is much more prevalent in medical information as it often discloses the most intimate details of a person that should not be shared. `When personally identifiable health information, for example, is disclosed to an employer, insurer, or family member, it can result in stigma, embarrassment, and discrimination'\cite{health-privacy}. This extends further since people may be reluctant to provide candid and complete disclosures of their sensitive information, even to physicians which may prevent a full diagnosis if their data isn't maintained in an anonymous fashion.

\subsection{Anonymity in AI}

We will focus directly on the specific use case of people's data in our application for automatic contouring for radiotherapy planning to narrow down the scope of anonymity.

The key takeaways from a publication regading use of clinical imaging data for AI in 2020 \cite{ethics-imaging-AI} can be summarised. `Sharing clinical data with outside entities is doesn't require specific customer consent as long as they are made aware of the ways their data may be used and as long as the external entities act as ethical data stweards'. Indeed, Royal Marsden Hospital mentions in their privacy policy that `Explicit consent may not be required if the information being used has been de-identified/anonymised. This means that it cannot be used to identify an individual person'\cite{royal-marsden-privacy-note}.

\section{UK law for medical data}

TODO

\section{How the data has been anonymised}

TODO

\section{Is the new tool a gold standard?}

When artificial intellgience tools are applied in the medical context there are several considerations that have to be discussed before the output of the model is taken as a final result. We should consider the worst-case-extreeme that the result of a radiotherpay planning application may mean for a patient. If an incorrect treatment area hilights a organ-at-risk (an organ which is at risk of being falsely subject to radiation) this may have life-threatening health implications for a patient. 

Accuracy of state-of-the-art imaging techniques are yet to reach 100\% \cite{}. Therefore the result of assisting technologies must not be taken as gossbel, but instead be treated with precaution, making sure to diligently check the correctness of the predictions.

\vspace{1em}

From \href{https://www.ncbi.nlm.nih.gov/pmc/articles/PMC7325854/pdf/main.pdf}{here} \cite{rise-of-ai-in-healthcare}

\begin{itemize}
    \item ``The healthcare ecosystem is realizing the importance of AI-powered tools in the next-generation healthcare technology''
    \item ``It is believed that AI can bring improvements to any process within healthcare operation and delivery'' like costs which may ``cut annual US healthcare costs by USD 150 billion in 2026''.
\end{itemize}

From \href{https://www.ncbi.nlm.nih.gov/pmc/articles/PMC6691444/pdf/JFMPC-8-2328.pdf}{here} \cite{overview-of-ai-medicine}

\begin{itemize}
    \item ``Radiology is the branch that has been the most upfront
    and welcoming to the use of new technology'' (6)
    \item ``AI could provide substantial aid in radiology by not
    only labeling abnormal exams but also by identifying quick negative
    exams in computed tomographies, X‑rays, magnetic resonance
    images especially in high volume settings, and in hospitals with
    less available human resources.''
    \item  ``A study conducted in 2016[17] found that physicians spent 27%
    of their office day on direct clinical face time with their patients
    and spent 49.2\% of their office day on electronic hospital records
    and desk work. When in the examination room with patients,
    physicians spent 52.9\% of their time on EHR and other work. In
    conclusion, the physicians who used documentation support such
    as dictation assistance or medical scribe services engaged in more
    direct face time with patients than those who did not use these
    services. In addition, increased AI usage in medicine not only
    reduces manual labor and frees up the primary care physician’s
    time but also increases productivity, precision, and efficacy''
\end{itemize}

From \href{https://onlinelibrary.wiley.com/doi/10.1002/cac2.12215}{here} \cite{AI-in-cancer-diagnosis-era}

\begin{itemize}
    \item ``Target volume and OAR delineation is a labor-intensive process, and its accuracy depends heavily on the experience of the radiation oncologists. CNN-based semantic segmentation has been consistently established as a state-of-the-art tool in the automated delineation''
    \item ``Given the variety of shapes, locations, and internal morphologies of tumors, automated contouring of tumor targets by DL is still a great challenge. Nonetheless, automatic contouring speeds up the process and improves consistency among radiation oncologists.''
    \item Generalizability and real-world application: ``More importantly, in clinical settings, to make a precise decision, oncologists need to consider a variety of data, including clinical manifestations, laboratory examinations, imaging data, and epidemiological histories. However, most recent studies have only adopted one type of data (such as imaging) as the input model. To mimic real clinical settings, a multimodal DL model incorporating the aforementioned information plus imaging data needs to be constructed in future studies.''
    \item Interpretability: the black-box problem: ``does not explain how the model generates outputs from given inputs. The large number of parameters involved makes it difficult for oncologists to understand how DL models analyze data and make decisions. ''.
    \item Data access and medical ethics: ``several ethical issues need to be addressed prior to the clinical implementation of DL tools. First, the degree of supervision required from physicians should be determined. Second, the responsible party for incorrect decisions made by DL tools should also be determined.''
\end{itemize}

\printbibliography

\end{document}