\documentclass[11pt]{article}

\usepackage[margin=1in]{geometry}
\usepackage{listings}
\usepackage{hyperref}
\usepackage[htt]{hyphenat}

\usepackage{biblatex}
\addbibresource{../../../source/bibliography.bib} %Import the bibliography file


\title{What is radiotherapy?}
\author{Anton Zhitomirsky}

\setlength{\parindent}{0pt}

\begin{document}

\maketitle

\section{Summary}

Clinicians use a system that uses high-resolution X-rays to produce contrasting images of cancerous tumours and surrounding soft tissue. Physicians then can target the cancerous tumour more precisely and decreasing radiation expsoure of healthy tissues \cite{radiotherapy-basic-concepts}.

Radiation therapy (radiotherapy) remains an improtant component of cancer treatment. In 2012 approximately 50\% of all cancer patients received radiation therapy, with an additional 40\% involving curative treatment \cite{radiotherapy-advances}.

\section{Cell Dealth}

Radiation-induced cell death is typically classified as interphase or proliferative death. Importantly, there is a consensus that cellular effects, including cell death, depend not only on radiation dose but cell type and a place in cell cycle \cite{cell-death}.

\subsection{Interphase death}

Cells cease to divide after radiation exposure and begin to die within hours. ``This is attributed to the damage of intercellular molecules and the activation of nuclease and proteolytic enzymes etc., after high doses of radiation, which leads to degradation [...]. Factors such as disruption of membrane structure and disorder of cell energy metabolism after irradiation are also important contributors to interphase death''\cite{cell-death}.

\subsection{Proliferative death}

Most cells undergo proliferative death after radiation. This results from ``mitotic catastrophe caused by accumulation of chromosomal aberrations and erroneous repair after radiation induces a DNA double-strand break [...] cells lose their ability to proliferate and begin to die.''\cite{cell-death}.

\section{Using Radiotherapy for Cancer}

Physicians use high-energy radiation to damage genetic material (deoxyribonucleic acid, DNA) of cells and thus block their ability to divide and proliferate further \cite{radiotherapy-advances}.

\section{The need for Radiotherapy}

% https://radiotherapy.org.uk/wp-content/uploads/2022/11/APPGRadiotherapy_Manifesto-Update_2022.pdf

'In England, waiting times are getting worse each month – over 1/3 of cancer patients (60,000) wait beyond the 62-day target, and 10,000 patients wait over 104 days.'

\printbibliography

\end{document}