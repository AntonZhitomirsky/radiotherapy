\documentclass[11pt]{article}
\usepackage[margin=1in]{geometry}
\usepackage{listings}

\title{Description of Source Data}
\author{Anton Zhitomirsky}

\begin{document}

\maketitle

\section{The data}

Results are structured in the file:

\begin{lstlisting}[language=bash]
/vol/biomedic3/bglocker/nnUNet
\end{lstlisting}

\begin{lstlisting}[language=inform]
-rwxr-xr-x   1 bglocker biomedia 236 Sep 24 15:16 exports
drwxr-sr-x   9 bglocker biomedia   9 Nov 25 10:55 nnUNet_preprocessed
drwxr-sr-x   9 bglocker biomedia  10 Nov 25 10:50 nnUNet_raw
drwxr-sr-x   9 bglocker biomedia   9 Nov 25 12:20 nnUNet_results
drwxr-sr-x  11 bglocker biomedia  11 Dec 16 09:10 nnUNet_testing
-rw-r--r--   1 bglocker biomedia 644 Oct 20 07:20 run_nnunet_0.sh
-rw-r--r--   1 bglocker biomedia 644 Oct 20 07:20 run_nnunet_1.sh
-rw-r--r--   1 bglocker biomedia 644 Oct 20 07:20 run_nnunet_2.sh
-rw-r--r--   1 bglocker biomedia 644 Oct 20 07:21 run_nnunet_3.sh
-rw-r--r--   1 bglocker biomedia 644 Oct 20 07:21 run_nnunet_4.sh
\end{lstlisting}

\section*{nnUNet\_raw}

nnUNet\_raw has the original (training) images with manual annotations. Each Dataset below is treated as a binary segmentation problem. See section \ref{section:itksnap}

\begin{lstlisting}[language=inform]
drwxr-sr-x  4 bglocker biomedia   5 Sep 17 13:47 Dataset001_Anorectum
drwxr-sr-x  3 bglocker biomedia   5 Sep 17 20:24 Dataset002_Bladder
drwxr-sr-x  3 bglocker biomedia   5 Sep 17 20:27 Dataset003_CTVn
drwxr-sr-x  3 bglocker biomedia   5 Sep 17 20:28 Dataset004_CTVp
drwxr-sr-x  3 bglocker biomedia   5 Sep 17 20:29 Dataset005_Parametrium
-rw-r--r--  1 bglocker biomedia 135 Nov 25 10:50 note
\end{lstlisting}

\subsection*{What is a Binary Segmentation Problem?}

\section{Viewing the Data using ItkSnap} \label{section:itksnap}

\end{document}