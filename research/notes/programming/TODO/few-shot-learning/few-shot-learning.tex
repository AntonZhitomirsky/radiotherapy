% Use https://builtin.com/machine-learning/few-shot-learning

% See universeg for sources

% Few-shot Semantic Segmentation. Few-shot (FS) meth-
% ods adapt to new tasks from few training examples, often
% by fine-tuning pretrained networks [ 22 , 75, 100 , 94 ]. Some
% few-shot semantic segmentation models generate predic-
% tions for new images (queries) containing unseen classes
% from just a few labeled examples (support) without addi-
% tional retraining. One strategy prevalent in both natural
% image [74 , 87 , 102 ] and medical image [ 19 , 59, 77, 91] FS
% segmentation methods is to employ large pre-trained models
% to extract deep features from the query and support images.
% These methods often involve learning meaningful prototyp-
% ical representations for each label [ 101 ]. Another medical
% FS segmentation strategy uses self-supervised learning to
% make up for the lack of training data and tasks [29 , 76]. In
% contrast to UniverSeg, these methods, focused on limited
% data regimes, tackle specific tasks involving generalizing to
% new classes in a particular subdomain, like abdominal CT or
% MRI scans [29, 76, 84, 98].
% In our work, we focus on avoiding any fine-tuning, even
% when given many examples for a new task, to avoid requir-
% ing the clinical or scientific user to have machine learning
% expertise and compute resources. Our proposed framework
% draws inspiration from ideas from some few-shot learning
% solutions, but aims to generalize to a universally broad set of
% anatomies, modalities, and datasets – even those completely
% unseen during training